\documentclass[compress]{beamer}
%,hyperref={pdfpagelabels=false}

\usepackage[ngerman,english]{babel}
\usepackage[T1]{fontenc}
\usepackage[latin1]{inputenc}
\usepackage{amsmath}
\usepackage{amsthm}
\usepackage{amsfonts}
\usepackage{helvet}
\usepackage{url}
\usepackage{listings}
\usepackage{xcolor}
\usepackage{color}
\usepackage{xspace} % Abstand hinter Variablennamen
\usepackage{fix-cm}
\usepackage{graphicx}
\usepackage{subfigure}
%\usepackage[square, sort, numbers, authoryear]{natbib}

\usepackage{beamerthemeLEA2}

%\bibliographystyle{plainnat}

\newcommand{\N}       {\mathbb{N}}          % natürliche Zahlen
\newcommand{\Z}       {\mathbb{Z}}          % ganze Zahlen
\newcommand{\R}       {\mathbb{R}}          % reelle Zahlen
\newcommand{\Prob}    {\mathrm{P}}          % Wahrscheinlichkeit
\newcommand{\inter}   {\cap}                % Schnittmenge
\newcommand{\union}   {\cup}                % Vereinigung
\newcommand{\Oh}      {O}                   % O-Notation (Landau-Symbole)
\newcommand{\mycite}[1]{\textcolor{tumgreen}{[#1]}} 

\newenvironment{changemargin}[2]{% 
  \begin{list}{}{%
    \setlength{\topsep}{0pt}%
    \setlength{\leftmargin}{#1}%
    \setlength{\rightmargin}{#2}%
    \setlength{\listparindent}{\parindent}%
    \setlength{\itemindent}{\parindent}%
    \setlength{\parsep}{\parskip}%
  }%
  \item[]}{\end{list}}
  
\definecolor{light-gray}{gray}{0.95}
\definecolor{middle-gray}{gray}{0.5}
  
\lstset{basicstyle=\footnotesize\ttfamily,breaklines=true}
\lstset{columns=fullflexible}
\lstset{framextopmargin=20pt}
\lstset{backgroundcolor=\color{white}}
\lstset{escapeinside={<@}{@>}}

\title{A Flexible Tool for Shape Analysis}
\subtitle{IDP Presentation}
\author{\href{emanuel.laude@in.tum.de}{Emanuel Laude}\\ \href{laehner@in.tum.de}{Zorah L\"ahner} } 
%\date{\today}
\date{January 16, 2015}
\institute{Technische Universit\"at M\"unchen}

% Inhaltsverzeichnis zu Begin von jedem Abschnitt einblenden?
%\AtBeginSection[]{
%  \begin{frame}
%    \frametitle{Outline}
%    \tableofcontents[currentsection]
%  \end{frame}
%}

\begin{document}

\begin{frame}
  \titlepage
\end{frame}

% Inhaltsverzeichnis
\begin{frame}
  \frametitle{Outline}
  \tableofcontents
\end{frame}

\section{Demo Time}

\begin{frame}
	\frametitle{Introduction}
	
	\bf{Demo Time}
	
\end{frame}

\section{Class Overview}

\begin{frame}
  \frametitle{Class Diagram: View}
  \begin{figure}[h]
	\centering
	\includegraphics[width=\textwidth]{diagram.pdf}
\end{figure}
\end{frame}

\begin{frame}
  \frametitle{Class Diagram: Domain}
  \begin{figure}[h]
	\centering
	\includegraphics[width=\textwidth]{diagram2.pdf}
\end{figure}
\end{frame}

\section{Customs}

\begin{frame}[c]
  \frametitle{Customs}
  \begin{center}
  
  \huge CustomContextMenuItem 
  
  \vspace{1em}
  
  or 
  
  \vspace{1em}
  
  CustomTab?
  \end{center}
\end{frame}

\begin{frame}[fragile]
  \frametitle{Example: VoronoiCellsContextMenuItem}
  
\begin{lstlisting}[language=C++, keywordstyle=\color{blue},
                stringstyle=\color{red},
                commentstyle=\color{green}, numbers=none]
<@\textcolor{middle-gray}{template<class T = Metric>}@>
class VoronoiCellsContextMenuItem : public CustomContextMenuItem {
<@\textcolor{middle-gray}{public:}@>
	<@\textcolor{middle-gray}{VoronoiCellsContextMenuItem<T>}@>
	<@\textcolor{middle-gray}{(shared\_ptr<Shape> shape, ShapeAnalyzerInterface* shapeAnalyzer)}@>
	 <@\textcolor{middle-gray}{: CustomContextMenuItem(shape, shapeAnalyzer) \{ \}}@>
    
    <@\textcolor{middle-gray}{virtual void onClick(vtkIdType pointId, vtkIdType faceId, QWidget* parent) \{}@>
        <@\textcolor{middle-gray}{(...)    \}}@>
<@\textcolor{middle-gray}{\};}@>
\end{lstlisting}
  
\end{frame}

\begin{frame}[fragile]
  \frametitle{Example: VoronoiCellsContextMenuItem}
  
\begin{lstlisting}[language=C++, keywordstyle=\color{blue},
                stringstyle=\color{red},
                commentstyle=\color{green}, numbers=none]
template<class T = Metric>
<@\textcolor{middle-gray}{class VoronoiCellsContextMenuItem : public CustomContextMenuItem \{}@>
<@\textcolor{middle-gray}{public:}@>
	<@\textcolor{middle-gray}{VoronoiCellsContextMenuItem<T>}@>
	<@\textcolor{middle-gray}{(shared\_ptr<Shape> shape, ShapeAnalyzerInterface* shapeAnalyzer)}@>
	 <@\textcolor{middle-gray}{: CustomContextMenuItem(shape, shapeAnalyzer) \{ \}}@>
    
    <@\textcolor{middle-gray}{virtual void onClick(vtkIdType pointId, vtkIdType faceId, QWidget* parent) \{}@>
        <@\textcolor{middle-gray}{(...)    \}}@>
<@\textcolor{middle-gray}{\};}@>
\end{lstlisting}
  
\end{frame}

\begin{frame}[fragile]
  \frametitle{Example: VoronoiCellsContextMenuItem}
  
\begin{lstlisting}[language=C++, keywordstyle=\color{blue},
                stringstyle=\color{red},
                commentstyle=\color{green}, numbers=none]
<@\textcolor{middle-gray}{template<class T = Metric>}@>
<@\textcolor{middle-gray}{class VoronoiCellsContextMenuItem : public CustomContextMenuItem \{}@>
<@\textcolor{middle-gray}{public:}@>
	VoronoiCellsContextMenuItem<T>
	 (shared_ptr<Shape> shape, ShapeAnalyzerInterface* shapeAnalyzer)
	 	: CustomContextMenuItem(shape, shapeAnalyzer) { }
    
    <@\textcolor{middle-gray}{virtual void onClick(vtkIdType pointId, vtkIdType faceId, QWidget* parent) \{}@>
        <@\textcolor{middle-gray}{(...)    \}}@>
<@\textcolor{middle-gray}{\};}@>
\end{lstlisting}
  
\end{frame}

\begin{frame}[fragile]
  \frametitle{Example: VoronoiCellsContextMenuItem}
  
\begin{lstlisting}[language=C++, keywordstyle=\color{blue},
                stringstyle=\color{red},
                commentstyle=\color{green}, numbers=none]
<@\textcolor{middle-gray}{template<class T = Metric>}@>
<@\textcolor{middle-gray}{class VoronoiCellsContextMenuItem : public CustomContextMenuItem \{}@>
<@\textcolor{middle-gray}{public:}@>
	<@\textcolor{middle-gray}{VoronoiCellsContextMenuItem<T>}@>
	<@\textcolor{middle-gray}{(shared\_ptr<Shape> shape, ShapeAnalyzerInterface* shapeAnalyzer)}@>
	 <@\textcolor{middle-gray}{: CustomContextMenuItem(shape, shapeAnalyzer) \{ \}}@>
    
    virtual void onClick(vtkIdType pointId, vtkIdType faceId, QWidget* parent) {
        (...)    }
<@\textcolor{middle-gray}{\};}@>
\end{lstlisting}
  
\end{frame}


\begin{frame}[fragile]
  \frametitle{Example: VoronoiCellsContextMenuItem}
  
\begin{lstlisting}[language=C++, keywordstyle=\color{blue},
                stringstyle=\color{red},
                commentstyle=\color{green}, numbers=none]
<@\textcolor{middle-gray}{virtual void onClick(...) \{}@>
    <@\textcolor{middle-gray}{bool ok;}@>
    vtkIdType source = QInputDialog::getInt(...);
    <@\textcolor{middle-gray}{if (!ok) \{ return; \}}@>
    <@\textcolor{middle-gray}{(...)}@>
    <@\textcolor{middle-gray}{if(ok) \{}@>
        <@\textcolor{middle-gray}{try \{}@>
            <@\textcolor{middle-gray}{auto m = make\_shared<T>(shape\_);}@>
            <@\textcolor{middle-gray}{auto fps = make\_shared<FarthestPointSampling>(shape\_, m, source, numberOfSegments);}@>
            <@\textcolor{middle-gray}{VoronoiCellSegmentation segmentation(shape\_, m, fps);}@>
            <@\textcolor{middle-gray}{shape\_->setColoring(segmentation.getSegments(), }@>
            	<@\textcolor{middle-gray}{Shape::Coloring::Type::PointSegmentation);}@>
        <@\textcolor{middle-gray}{\} catch(metric::MetricError\& e) \{}@>
            <@\textcolor{middle-gray}{QMessageBox::warning(parent, "Exception", e.what());}@>
        <@\textcolor{middle-gray}{\}
    \}
\}}@>
\end{lstlisting}
  
\end{frame}

\begin{frame}[fragile]
  \frametitle{Example: VoronoiCellsContextMenuItem}
  
\begin{lstlisting}[language=C++, keywordstyle=\color{blue},
                stringstyle=\color{red},
                commentstyle=\color{green}, numbers=none]
<@\textcolor{middle-gray}{virtual void onClick(...) \{}@>
    <@\textcolor{middle-gray}{bool ok;}@>
    <@\textcolor{middle-gray}{vtkIdType source = QInputDialog::getInt(...);}@>
    <@\textcolor{middle-gray}{if (!ok) \{ return; \}}@>
    <@\textcolor{middle-gray}{(...)}@>
    <@\textcolor{middle-gray}{if(ok) \{}@>
        <@\textcolor{middle-gray}{try \{}@>
            auto m = make_shared<T>(shape_);
            auto fps = make_shared<FarthestPointSampling>(shape_, m, source, numberOfSegments);
            VoronoiCellSegmentation segmentation(shape\_, m, fps);
            <@\textcolor{middle-gray}{shape\_->setColoring(segmentation.getSegments(), }@>
            	<@\textcolor{middle-gray}{Shape::Coloring::Type::PointSegmentation);}@>
        <@\textcolor{middle-gray}{\} catch(metric::MetricError\& e) \{}@>
            <@\textcolor{middle-gray}{QMessageBox::warning(parent, "Exception", e.what());}@>
        <@\textcolor{middle-gray}{\}
    \}
\}}@>
\end{lstlisting}
  
\end{frame}

\begin{frame}[fragile]
  \frametitle{Example: VoronoiCellsContextMenuItem}
  
\begin{lstlisting}[language=C++, keywordstyle=\color{blue},
                stringstyle=\color{red},
                commentstyle=\color{green}, numbers=none]
<@\textcolor{middle-gray}{virtual void onClick(...) \{}@>
    <@\textcolor{middle-gray}{bool ok;}@>
    <@\textcolor{middle-gray}{vtkIdType source = QInputDialog::getInt(...);}@>
    <@\textcolor{middle-gray}{if (!ok) \{ return; \}}@>
    <@\textcolor{middle-gray}{(...)}@>
    <@\textcolor{middle-gray}{if(ok) \{}@>
        <@\textcolor{middle-gray}{try \{}@>
            <@\textcolor{middle-gray}{auto m = make\_shared<T>(shape\_);}@>
            <@\textcolor{middle-gray}{auto fps = make\_shared<FarthestPointSampling>(shape\_, m, source, numberOfSegments);}@>
            <@\textcolor{middle-gray}{VoronoiCellSegmentation segmentation(shape\_, m, fps);}@>
            shape_->setColoring(segmentation.getSegments(), 
            	Shape::Coloring::Type::PointSegmentation);
        <@\textcolor{middle-gray}{\} catch(metric::MetricError\& e) \{}@>
            <@\textcolor{middle-gray}{QMessageBox::warning(parent, "Exception", e.what());}@>
        <@\textcolor{middle-gray}{\}
    \}
\}}@>
\end{lstlisting}

\end{frame}

\begin{frame}
	\frametitle{Shape Coloring}
	
	\begin{columns}
    		\begin{column}{.4\linewidth}
      		\begin{itemize}
			\item<1> PointSegmentation
			\item<1> FaceSegmentation
			\item<2> PointRGB
			\item<2> FaceRGB
			\item<3> PointScalar
			\item<3> FaceScalar
		\end{itemize}
   		\end{column}
    		\begin{column}{.6\linewidth}
		
			 \begin{figure}[h]
				\centering
				\only<1-1>{\includegraphics[width=\textwidth]{segmentation.png}}
				\only<2-2>{\includegraphics[width=\textwidth]<2>{rgb.png}}
				\only<3-3>{\includegraphics[width=\textwidth]<3>{scalar.png}}
			\end{figure}
      
    		\end{column}
  	\end{columns}
	
\end{frame}

\begin{frame}
	\frametitle{CustomTab}
	
	\begin{itemize}
		\item Constructor signature:
		\begin{itemize}
			\item \texttt{HashMap<vtkActor*, shared\_ptr<Shape>>\& \textcolor{blue}{shapes}}
			\item \texttt{HashMap<shared\_ptr<PointCorrespondence>, bool>\& \textcolor{blue}{pointCorrespondences}}
			\item \texttt{HashMap<shared\_ptr<FaceCorrespondence>, bool>\& \textcolor{blue}{faceCorrespondences}}
			\item \texttt{ShapeAnalyzerInterface* \textcolor{blue}{shapeAnalyzer}}
		\end{itemize}
		\item Abstract functions:
		\begin{itemize}
			\item \texttt{\textcolor{blue}{void} onShapeAdd(Shape* shape)}
			\item \texttt{\textcolor{blue}{void} onShapeEdit(Shape* shape)}
			\item \texttt{\textcolor{blue}{void} onShapeDelete(Shape* shape)}
			\item \texttt{\textcolor{blue}{void} onClear()}
		\end{itemize}
	\end{itemize}
\end{frame}

\begin{frame}[fragile]
\frametitle{Connect Custom Classes}

\lstset{language=C++,
                keywordstyle=\color{blue},
                stringstyle=\color{red},
                commentstyle=\color{green},
                morecomment=[l][\color{magenta}]{\#},
                numbers=none
}
\begin{lstlisting}
template<class T, class... Args>
class Factory { ... }
<@\textcolor{middle-gray}{typedef Factory<CustomContextMenuItem, shared\_ptr<Shape>, ShapeAnalyzerInterface*> CustomContextMenuItemFactory;}@>

<@\textcolor{middle-gray}{typedef Factory<CustomTab, const HashMap<vtkActor*>, shared\_ptr<Shape>>\&,}@>
    <@\textcolor{middle-gray}{const HashMap<shared\_ptr<PointCorrespondence>, bool>\&,}@>
    <@\textcolor{middle-gray}{const HashMap<shared\_ptr<FaceCorrespondence>, bool>\&,}@>
    <@\textcolor{middle-gray}{ShapeAnalyzerInterface*> CustomTabFactory;}@>
\end{lstlisting}

\end{frame}

\begin{frame}[fragile]
\frametitle{Connect Custom Classes}

\lstset{language=C++,
                keywordstyle=\color{blue},
                stringstyle=\color{red},
                commentstyle=\color{green},
                morecomment=[l][\color{magenta}]{\#},
                numbers=none
}
\begin{lstlisting}
<@\textcolor{middle-gray}{template<class T, class... Args>}@>
<@\textcolor{middle-gray}{class Factory \{ ... \}}@>

typedef Factory<CustomContextMenuItem, shared_ptr<Shape>, ShapeAnalyzerInterface*> CustomContextMenuItemFactory;}@>

typedef Factory<CustomTab, const HashMap<vtkActor*,
    shared_ptr<Shape>>&,}@>
    const HashMap<shared_ptr<PointCorrespondence>, bool>&,
    const HashMap<shared_ptr<FaceCorrespondence>, bool>&,
    ShapeAnalyzerInterface*> CustomTabFactory;
\end{lstlisting}

\end{frame}

\begin{frame}[fragile]
\frametitle{Connect Custom Classes}

\lstset{language=C++,
                keywordstyle=\color{blue},
                stringstyle=\color{red},
                commentstyle=\color{green},
                morecomment=[l][\color{magenta}]{\#},
                numbers=none
}
\begin{lstlisting}
struct CustomClassesRegistry {
    static void registerContextMenuItems() {
       CustomContextMenuItemFactory::getInstance()->Register<MyMenuItemClass>("key", "My Menu Item Label");
    }
    static void registerTabs() {
        CustomTabFactory::getInstance()->Register<MyTabClass>("key", "My Tab Label");
    }
}
\end{lstlisting}
  
\end{frame}


\begin{frame}[fragile]
\frametitle{Connect Custom Menu Items}
\lstset{language=C++,
                keywordstyle=\color{blue},
                stringstyle=\color{red},
                commentstyle=\color{green},
                morecomment=[l][\color{magenta}]{\#},
                numbers=none
}
\begin{lstlisting}
static void registerContextMenuItems() {
    CustomContextMenuItemFactory::getInstance()->Register<ExtractSegmentContextMenuItem>("extract_segment",
        "This>>is>>my>>menu path>>My item 1");
   
    CustomContextMenuItemFactory::getInstance()->Register<VoronoiCellsContextMenuItem<GeodesicMetric>>("voronoicells_geodesic",
        "This>>is>>My item 2");
}     
\end{lstlisting}
  \begin{figure}[h]
	\centering
	\includegraphics[width=0.60\textwidth]{menu.png}
\end{figure}
\end{frame}

\begin{frame}[fragile]
\frametitle{Connect Custom Tabs}
\lstset{language=C++,
                keywordstyle=\color{blue},
                stringstyle=\color{red},
                commentstyle=\color{green},
                morecomment=[l][\color{magenta}]{\#},
                numbers=none
}
\begin{lstlisting}
static void registerTabs() {
    CustomTabFactory::getInstance()->Register<IdentityMatchingTab>("identity_matching", "Shapes>>My Shape Tab 1");

    CustomTabFactory::getInstance()->Register<ShapeInterpolationTab>("shape_interpolation", "Correspondences>>My Correspondences Tab 1");
}
\end{lstlisting}
  \begin{figure}[h]
	\centering
	\includegraphics[width=\textwidth]{tabs.png}
\end{figure}
\end{frame}

\section{PETSc and SLEPc}
\begin{frame}
\frametitle{PETSc and SLEPc}
\begin{itemize}
	\item PETSc: (Sparse) linear algebra (e.g. matrix vector multiplication, linear systems, ...)
	\item SLEPc: (Sparse) eigenvalue solver (builds up on PETSc)
	\item Heavily used in the "Computational Science and Engineering" community for solving PDEs (e.g. Computational Fluid Dynamics)
	\item Based on MPI (therefore well suited for distributed HPCs)
	\item Can also be used with CUDA and cuBLAS (instead of MPI).
	\item Drawback: Coding is kinda' low level :(
\end{itemize}
\end{frame}

\begin{frame}[fragile]
\frametitle{A Flavor of PETSc}
$$
	\mathbf{b} := \mathbf{A} \cdot \mathbf{a}
$$
\lstset{language=C++,
                keywordstyle=\color{blue},
                stringstyle=\color{red},
                commentstyle=\color{green},
                morecomment=[l][\color{magenta}]{\#},
                numbers=none
}
\begin{lstlisting}
    // Initialize matrix A
    int n = 10;
    int m = 15;
    Mat A;
    MatCreateSeqDense(PETSC_COMM_SELF, n, m, NULL, &A);
    for(PetscInt i = 0; i < n; i++) {
        for(PetscInt j = 0; j < m; j++) {
            PetscReal aij = rand();
            MatSetValue(A, i, j, aij, INSERT_VALUES);
        }
    }
    MatAssemblyBegin(A, MAT_FINAL_ASSEMBLY);
    MatAssemblyEnd(A, MAT_FINAL_ASSEMBLY);
\end{lstlisting}

\end{frame}

\begin{frame}[fragile]
\frametitle{A Flavor of PETSc}
$$
	\mathbf{b} := \mathbf{A} \cdot \mathbf{a}
$$
\lstset{language=C++,
                keywordstyle=\color{blue},
                stringstyle=\color{red},
                commentstyle=\color{green},
                morecomment=[l][\color{magenta}]{\#},
                numbers=none
}
\begin{lstlisting}
    //Initialization of the vectors a and b
    Vec a, b;
    MatGetVecs(A, &a, &b);
    for(PetscInt i = 0; i < size; i++) {
        VecSetValue(a, i, rand(), INSERT_VALUES);
    }
    VecAssemblyBegin(a);
    VecAssemblyEnd(a);
\end{lstlisting}

\end{frame}

\begin{frame}[fragile]
\frametitle{A Flavor of PETSc}
$$
	\mathbf{b} := \mathbf{A} \cdot \mathbf{a}
$$
\lstset{language=C++,
                keywordstyle=\color{blue},
                stringstyle=\color{red},
                commentstyle=\color{green},
                morecomment=[l][\color{magenta}]{\#},
                numbers=none
}
\begin{lstlisting}
    // Actual computation
    MatMult(A, a, b);
    VecView(b, PETSC_VIEWER_STDOUT_SELF);
    MatDestroy(&A);
    VecDestroy(&a);
    VecDestroy(&b);
\end{lstlisting}

\end{frame}

\begin{frame}[fragile]
\frametitle{A Flavor of PETSc and CUDA}
$$
	\mathbf{x} = a \cdot \mathbf{x} + \mathbf{y}
$$
\lstset{language=C++,
                keywordstyle=\color{blue},
                stringstyle=\color{red},
                commentstyle=\color{green},
                morecomment=[l][\color{magenta}]{\#},
                numbers=none
}
\begin{lstlisting}
ierr = VecCUDACopyToGPU(xin);CHKERRQ(ierr);
ierr = VecCUDACopyToGPU(yin);CHKERRQ(ierr);
try {
    cusp::blas::axpy(*((Vec_CUDA*)xin->spptr)->GPUarray,
        *((Vec_CUDA*)yin->spptr)->GPUarray,
        alpha
    );
    yin->valid_GPU_array = PETSC_CUDA_GPU;
    ierr = WaitForGPU();
    CHKERRCUDA(ierr);
} catch(char *ex) { /*...*/ }
\end{lstlisting}

\end{frame}

\section{A Function Transfer Tab}

\begin{frame}
\frametitle{Functional maps: Intro}
Goal: Transfer a function $f:V_M \to \mathbb{R}$ from shape $M$ to shape $N$. (Please note: $M = (V_M:=\{v_1, v_2, \dots, v_n\}, T_M)$ and therefore $f \in \mathbb{R}^n$)\\
Assumption: $M$ and $N$ are related by an \emph{isometry} \\
(Ovsjanikov et al. 2012)
  \begin{figure}[h]
	\centering
	\includegraphics[width=0.6\textwidth]{fm.pdf}
\end{figure}
\end{frame}

\begin{frame}[fragile]
\frametitle{Functional maps: ONB}
$$
	f = \sum_{i=1}^{n} a_i\phi_i \in \mathbb{R}^n
$$
First guess (Canonical basis of $\mathbb{R}^n$): $\phi_i := \delta_{ij}$ \\

\begin{block}{Question}
Is this really a good idea? Answer: Nope! \\
Transferring a function $f$ from $M$ to $N$ requires the correspondence Matrix $C\in \mathbb{R}^{n \times n}$ (But that is what we are looking for)
\end{block}
\end{frame}

\begin{frame}[fragile]
\frametitle{Functional maps: ONB}
Better choice:
Eigenvectors $\{\phi^M_i\}_{i=1}^n \subset \mathbb{R}^n$ and $\{\phi^N_i\}_{i=1}^n \subset \mathbb{R}^n$ of the Laplace-Beltrami operators $\Delta_M=M_M^{-1} \cdot L_M \in \mathbb{R}^{n \times n}$ and $\Delta_N=M_N^{-1} \cdot L_N \in \mathbb{R}^{n \times n}$
\begin{figure}[h]
	\centering
	\includegraphics[width=1.0\textwidth]{Michael.eps}
\end{figure}
  \begin{figure}[h]
	\centering
	\includegraphics[width=1.0\textwidth]{Michael1.eps}
\end{figure}

\end{frame}

\begin{frame}[fragile]
\frametitle{Functional maps: ONB}
\begin{block}{Property 1}
If two shapes $M, N$ are related by a perfect isometry, their Laplace-Beltrami eigenvectors are the same up two sign flip.
\end{block}
\begin{block}{Property 2}
If the signal $f$ is sufficiently smooth we can represent it accurately (enough) by the first $20$ eigenvectors:
$$
	f \approx \sum_{i=1}^{20} a_i\phi_i \in \mathbb{R}^n
$$
\end{block}
\end{frame}

\begin{frame}[fragile]
\frametitle{Functional maps: ONB}
\begin{block}{Idea}
Transfer $f$ from $M$ to $N$ by projecting it onto Laplace-Beltrami eigenbasis $\{\phi_i^M\}_{i=1}^{20}$ and flip signs of coefficients accordingly:
$$
	 \langle \phi_i^M, f \rangle_{M_M} = \langle \phi_i^M, \sum_{j=1}^{20} a_j\phi_j^M \rangle_{M_M} = \sum_{j=1}^{20} a_j \langle \phi_i^M, \phi_j^M  \rangle_{M_M} = a_i 
$$
$$
	t(f) = \sum_{i=1}^{20} b_i \phi_i^N = \sum_{i=1}^{20} \underbrace{c_{ii} a_i}_{b_i} \phi_i^N \quad \text{with } c_{ii} \in \{-1, 1\} 
$$
$$
	t(f) = \underbrace{C \cdot a}_{=b} \cdot \Phi^N \quad \text{with } C \in \mathbb{R}^{20 \times 20}, C \text{ diagonal} 
$$
\end{block}
\end{frame}

\begin{frame}[fragile]
\frametitle{Functional maps: Idea}
Caveat: Since we never deal with perfect isometries, $C$ is not perfectly diagonal:
\begin{figure}[h]
	\centering
	\includegraphics[width=.3\textwidth]{C.png}
\end{figure}
\end{frame}

\begin{frame}[fragile]
\frametitle{Functional maps: Compute C}
Assume we are given a set of corresponding constraint functions represented as 20-dimensional vectors of coefficients $a_i^\top,b_i^\top \in \mathbb{R}^{20}$:
\begin{figure}[htp]
  \begin{center}
    \subfigure{\includegraphics[scale=0.11]{heat_bump50.pdf}}
    \subfigure{\includegraphics[scale=0.11]{heat_bump200.pdf}} 
    \subfigure{\includegraphics[scale=0.11]{wave_kernel.pdf}}
  \end{center}
\end{figure}
$$
	\min_C \underbrace{\frac{1}{2} \| B - CA \|_F^2}_{\text{data term}} + \lambda \underbrace{|W \circ C|_1}_{\text{regularizer}}
$$
\end{frame}
\begin{frame}[fragile]
\frametitle{Functional maps: Define W}
First guess: Squared distance from diagonal: $W_{ij}:=10 \cdot (i-j)^2$\\
Better:
$$
	W_{ij} :=\min\{|i - j|, 1\} \cdot (100 - \left(\frac{\min\{i, j\}}{20}\right)^3 \cdot 100)
$$
\begin{figure}[htp]
  \begin{center}
    \subfigure{\includegraphics[scale=0.15]{W_1.eps}}
    \subfigure{\includegraphics[scale=0.15]{W_2.eps}} 
    \subfigure{\includegraphics[scale=0.15]{W_3.eps}}
  \end{center}
\end{figure}
\end{frame}

\begin{frame}[fragile]
\frametitle{Functional maps: Outliers}
$$
	\min_{C,O} \frac{1}{2} \| B - CA -O \|_F^2 + \lambda |W \circ C|_1 + \mu \|O\|_{2,1}
$$
where $\|O\|_{2,1}:= \sum_{i=1}^l \|o_i^\top \|_2$ (Pokrass et al. 2013)
\end{frame}

\begin{frame}[fragile]
\frametitle{Functional maps: Optimization}
$$
	\min_{C,O} \underbrace{\frac{1}{2} \| B - CA -O \|_F^2}_{\text{differentiable}} +\underbrace{\lambda |W \circ C|_1 + \mu \|O\|_{2,1}}_{\text{not differentiable}}
$$
\begin{itemize}
\item Whole function convex
\item Optimization by Forward-Backward Splitting (Generalized projected gradient descent)
\end{itemize}
\end{frame}

\begin{frame}[fragile]
\frametitle{A Function Transfer Tab}
\bf{Demo time}
\end{frame}

\section*{}

\begin{frame}
\frametitle{Questions?}
\bf{Thank you for your attention!}
\end{frame}

\end{document}
