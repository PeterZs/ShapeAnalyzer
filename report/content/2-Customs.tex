% !TeX encoding=utf8
% !TeX spellcheck = en-US

\chapter{Customs}

\section{Exceptions}
\label{sec:Exceptions}

There are two types of exceptions you might want to throw: Exceptions (\ref{subsec:Exceptions}) and Errors (\ref{subsec:Errors}). The first one is kind of optional, it includes all normal exceptions C++ provides, the second one is for non-fatal problems that might occur and that can actually be handled.

\subsection{Errors}
\label{subsec:Errors}


\subsection{Exceptions}
\label{subsec:Exceptions}

\begin{itemize}
	\item \textbf{Throw when} you have special cases that are due to previous faulty programming or unexpected errors
	\item \textbf{Documentation} not necessary
\end{itemize}

Any exception will be caught by the main program and showed within an error message. The program will then terminate. In order to make debugging easier a short description of the problem is useful.

\begin{lstlisting}[style=lstStyleCpp]
if(s >= points_->size()) {
        throw invalid_argument("Source point (" + to_string(s) + ") 
        larger than number of points (" + to_string(points_->size()) + ") 
        in " + __PRETTY_FUNCTION__);
}
\end{lstlisting}