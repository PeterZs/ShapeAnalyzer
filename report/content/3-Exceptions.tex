% !TeX encoding=utf8
% !TeX spellcheck = en-US

\chapter{Exceptions}

There are two types of exceptions you might want to throw: Exceptions (\ref{sec:Exceptions}) and Errors (\ref{sec:Errors}). The first one is kind of optional, it includes all normal exceptions C++ provides, the second one is for non-fatal problems that might occur and that can actually be handled. In both cases do not use the \texttt{throws} keyword in the function signature.

When you are using VTK classes you can use the ErrorObserver (\ref{sec:ErrorObserver}) to handle VTK specific exceptions.

\section{Exceptions}
\label{sec:Exceptions}

\begin{itemize}
	\item \textbf{Throw when} you have special cases that are due to previous faulty programming or unexpected errors
	\item \textbf{Documentation} not necessary
	\item \textbf{Found in} C++ std library
\end{itemize}

Any exception will be caught by the main program and shown within an error message. The program will then terminate. In order to make debugging easier a short description of the problem and the function name is useful.

\begin{lstlisting}[style=lstStyleCpp]
if(s >= points_->size()) {
        throw invalid_argument("Source point (" + to_string(s) + ") 
        larger than number of points (" + to_string(points_->size()) + ") 
        in " + __PRETTY_FUNCTION__);
}
\end{lstlisting}


\section{Errors}
\label{sec:Errors}

\begin{itemize}
	\item \textbf{Throw when} you notice faulty data or the like that can not be processed, but it is not necessary to terminate the whole program
	\item \textbf{Documentation} use the \texttt{@throws} command and define the Error class 
	\item \textbf{Found in} util or the namespaces folder
\end{itemize}

Error is a ShapeAnalyzer specific class that inherits from exception but does not add any functionality. Every namespace has its own Error class that is named \texttt{<namespace>Error}. This allows more specific error handling by catching the different classes. 

If your exception is not due to faulty programming logic or you think it can be solved by a higher level class, use this classes instead of a std exception. This can be for example detecting a mesh with borders but your algorithm does not support borders. Most of the time this means you terminate your algorithm without a result and let the user know about the problem with an pop-up. Again a short description of the problem and the function name is useful. You can use \texttt{QMessageBox::warning} for the pop-up but you must give the QMessageBox the ShapeAnalyzer as the parent widget, so catch the Error in your Tab or ContextMenuItem (see \ref{sec:Custom}) and create the QMessageBox there.  
\begin{mdframed}
It is possible to give null as the parent widget but you must not do that. It might cause unwanted behavior like the QMessageBox not appearing in the foreground.
\end{mdframed}

If your function is throwing an Error use the \texttt{@throws} or \texttt{@exception} keyword in the Doxygen documentation. Give the class of the error and a short description of what might cause this error. 

\begin{lstlisting}[style=lstStyleCpp]
if(ierr != 0) {
    throw LaplaceBeltramiError(string(
        "Error occured in computation of Laplace-Beltrami eigenfunction 
	(Probably due to corrupted shape)")
	.append(" in ").append(__PRETTY_FUNCTION__));
}
\end{lstlisting}

\begin{lstlisting}[style=lstStyleCpp]
try {
    PetscFEMLaplaceBeltramiOperator laplacian(shape_, 100);
    vtkSmartPointer<vtkDoubleArray> eigenfunction 
    		= laplacian.getEigenfunction(i);
} catch(LaplaceBeltramiError& e) {
    QMessageBox::warning(parent, "Exception", e.what());
}
\end{lstlisting}

\begin{mdframed}
Some namespaces do not have their own Error class yet. If you need it, just create it. You can use \texttt{MetricError} as a template. 
\end{mdframed}

\section{ErrorObserver}
\label{sec:ErrorObserver}

\begin{itemize}
	\item \textbf{Use when} you want to handle VTK specific errors
	\item \textbf{Found in} util
\end{itemize}

VTK classes in general do not throw exceptions instead they write error and warning messages to observers. If you want to handle those problems use the ErrorObserver class.

\begin{lstlisting}[style=lstStyleCpp]
// vtk object that we want to observe
vtkSmartPointer<vtkTriangleFilter> triangleFilter 
	= vtkSmartPointer<vtkTriangleFilter>::New();
// new ErrorObserver object
vtkSmartPointer<ErrorObserver> triangleObserver 
	= vtkSmartPointer<ErrorObserver>::New();

// list the observer to the triangle filter
// if you are not interested in warnings leave the second line out
triangleFilter->AddObserver(vtkCommand::ErrorEvent, triangleObserver);
triangleFilter->AddObserver(vtkCommand::WarningEvent, triangleObserver);

// do the things you want to do with your vtk object
triangleFilter->SetInputConnection(input);
triangleFilter->Update();

// see if messages were written into the observer and handle that
if (triangleObserver->GetError()) {
        showErrorMessage("The file cound not be opended", 
        				      triangleObserver->GetErrorMessage());
        return;
}
if (triangleObserver->GetWarning()) {
        showErrorMessage("There was a warning reading the file",
        				       triangleObserver->GetWarningMessage());
}
\end{lstlisting}

Notice that if you use the same ErrorObserver for several objects or call several functions before checking the messages, you will only receive the latest message. 
